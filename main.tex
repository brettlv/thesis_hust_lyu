%%
%% This is file `hustthesis-zh-example.tex',
%% generated with the docstrip utility.
%%
%% The original source files were:
%%
%% hustthesis.dtx  (with options: `example-zh')
%% 
%% This is a generated file.
%% 
%% Copyright (C) 2013-2014 by Xu Cheng <xucheng@me.com>
%%               2014-2016 by hust-latex <https://github.com/hust-latex>
%% 
%% This work may be distributed and/or modified under the
%% conditions of the LaTeX Project Public License, either version 1.3
%% of this license or (at your option) any later version.
%% The latest version of this license is in
%%   http://www.latex-project.org/lppl.txt
%% and version 1.3 or later is part of all distributions of LaTeX
%% version 2005/12/01 or later.
%% 
%% This work has the LPPL maintenance status `maintained'.
%% 
%% The Current Maintainer of this work is hust-latex Organization.
%% 
%% This work consists of the files hustthesis.bst, hustthesis.dtx,
%% hustthesis.ins and the derived file hustthesis.cls
%% along with its document and example files.
%% 
%% \CharacterTable
%% {Upper-case    \A\B\C\D\E\F\G\H\I\J\K\L\M\N\O\P\Q\R\S\T\U\V\W\X\Y\Z
%%  Lower-case    \a\b\c\d\e\f\g\h\i\j\k\l\m\n\o\p\q\r\s\t\u\v\w\x\y\z
%%  Digits        \0\1\2\3\4\5\6\7\8\9
%%  Exclamation   \!     Double quote  \"     Hash (number) \#
%%  Dollar        \$     Percent       \%     Ampersand     \&
%%  Acute accent  \'     Left paren    \(     Right paren   \)
%%  Asterisk      \*     Plus          \+     Comma         \,
%%  Minus         \-     Point         \.     Solidus       \/
%%  Colon         \:     Semicolon     \;     Less than     \<
%%  Equals        \=     Greater than  \>     Question mark \?
%%  Commercial at \@     Left bracket  \[     Backslash     \\
%%  Right bracket \]     Circumflex    \^     Underscore    \_
%%  Grave accent  \`     Left brace    \{     Vertical bar  \|
%%  Right brace   \}     Tilde         \~}

%% https://github.com/hust-latex/hustthesis

\documentclass[format=draft,language=chinese,degree=phd]{hustthesis}
%\usepackage{ctex}
%\usepackage{xeCJK}
%\setCJKmainfont{SimSun.ttf}
%\setCJKsansfont{SimHei.ttf}
%\setCJKmonofont{SimFang.ttf}
\usepackage{verbatim}
\usepackage{fontspec}


\stuno{D201880056}
\schoolcode{10487}
%\title{\LaTeX 模板使用示例}{An Example of Using hustthesis \LaTeX{} Template}
\title{活动星系核的能谱演化与时变研究}{AGN's spectrum evolution and timing study }
\author
{吕兵}{Lyu Bing}
\major{理论物理}{Theoretical Physics}
%{物理学院天文学系}{School of Physics, Department of Astronomy}
\supervisor
{吴庆文\hspace{1em}教授}{Prof. Wu Qingwen}
\date{2021}{7}{30} 
\clearpage
\zhabstract{
 活动星系核是宇宙中极亮的一种天体源,

}
\zhkeywords{活动星系核}

\enabstract
{
%active galactic nucleus
Active galactic nuclei (AGNs)


}
\enkeywords
{AGN}

\graphicspath{{./}{fig/}}

\begin{document}
\frontmatter
\maketitle
\makeabstract
\tableofcontents
%\listoffigures
%\listoftables
\mainmatter

\chapter{引言}\label{chapter:1}
\section{活动星系核观测历史与分类}\label{sec:1.1}
活动星系核的发现最早是由进行的\footnote{\label{footnote:1}脚注}

\section{活动星系核统一模型与挑战}\label{sec:1.2}
\section{变脸活动星系核}\label{sec:1.3}
%\subsection{分类}\label{sec:2}
%\subsubsection{统一模型}\label{sec:3}





\chapter{变脸活动星系核的能谱演化}\label{chapter:2}
\section{变脸活动星系核}\label{sec:2.1}
\section{变脸活动星系核}\label{sec:2.2}
\section{变脸活动星系核}\label{sec:2.3}



\chapter{变脸活动星系核的尘埃反响映射}\label{chapter:3}
\section{变脸活动星系核}\label{sec:3.1}
\section{变脸活动星系核}\label{sec:3.2}
\section{变脸活动星系核}\label{sec:3.3}



\chapter{总结与展望}\label{chapter:last}
\section{变脸活动星系核}\label{sec:last.1}
\section{变脸活动星系核}\label{sec:last.2}
\section{变脸活动星系核}\label{sec:last.3}


\chapter{字体}\label{chapter:n}
\section{字体}
\textit{$N_\mathrm{H}$}
普通\textbf{粗体}\textit{斜体}\emph{斜体}
\hei{黑体}\kai{楷体}\fangsong{仿宋}
 \textbf {加粗命令}
显示直立文本: \textup{文本}
意大利斜体: \textit{文本} 
slanted斜体: \textsl{文本}
显示小体大写文本: \textsc{文本}
中等权重: \textmd{文本}
加粗命令: \textbf {文本}
默认值: \textnormal{文本}
斜体字:\textit{italic},或者\emph{italic}
使用等宽字体:\texttt{code}
使用无衬线字体:\textsf{sans-serif}
所有字母大写:\uppercase{CAPITALS}
所有字母大写,但小写字母比较小:\textsc{Small Capitals}
\section{参考文献示例}
这是一篇中文参考文献\cite{TEXGURU99};这是一篇英文参考文献\cite{knuth};同时引用\cite{TEXGURU99,knuth}。


\bibliography{ref-example.bib}
\clearpage
%\begin{publications}  
	%\begin{itemize}
	%\item 论文1 写写
	%\item 论文2 例子
	%\end{itemize}
%\end{publications}


%\chapter{这是一个附录}\label{appendix:1}
%附录正文。
\appendix
\chapter{攻读博士学位期间发表的论文}%\label{appendix:1}
%附录正文。
\begin{enumerate}[labelindent=0pt,label={[\arabic*]},itemsep=0.5ex]
	\fontsize{10.5pt}{10.5pt}\selectfont
	%\begin{itemize}
	\item 论文1
	\item 论文2	
	%\end{itemize}
\end{enumerate}




\backmatter
\begin{ack}
	致谢正文。
\end{ack}


\end{document}
\endinput
%%
%% End of file `hustthesis-zh-example.tex'.